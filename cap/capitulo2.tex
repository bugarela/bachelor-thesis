\chapter{TLA\textsuperscript{+}}
\label{cap2}

TLA+ é uma linguagem de especificação de software, criada por Leslie Lamport [CITAR] voltada à modelagem de sistemas concorrentes. Ela se propõe a oferecer uma maneira mais simples de escrever um algoritmo, ao utilizar um nível de abstração acima do que há ao escrever código em uma linguagem de programação. Assim, ao programar, não é necessário atentar-se a detalhes de implementação, permitindo o foco no comportamento do algoritmo - e não das suas dependências.

As especificações são descritas em fórmulas matemáticas, com pequenas adaptações de sintaxe. Para facilitar a curva de aprendizado para engenheiros, foi criada a linguagem PlusCal, com uma sintaxe semelhante a linguagens de programação imperativas, e que traduz seus programas para TLA+. A linguagem PlusCal não permite especificar sistemas tão complexos quanto os que podem ser escritos diretamente em TLA+, mas, devido à tradução para a linguagem original, aproveita completamente as capacidades dela de verificação de propriedades.

O método de especificação é baseado em máquinas de estados e, sendo assim, a descrição de um modelo é composta por uma condição inicial, que determina os possíveis estados inciais, e por uma relação de transições, que determina os possíveis estados que podem suceder cada estado em uma execução. Dessa forma, o conjunto de comportamentos especificado é composto por todos os comportamentos cujo estado inicial satisfaz a condição inicial e todas as transições estão na relação.

Lamport destaca [HYPERBOOK] que as especificações deveriam ser sobre modelos de uma abstração do sistema, e não algo retirado do próprio sistema. Semelhante à planta de um edifício, a especificação pode ser consultada para obter informações sobre o edifício (ou programa) de forma mais conveniente, além de ser capaz de facilitar uma série de verificações e perceber problemas enquanto a mudança ainda não é inviavelmente custosa.

\input{JarrosDeAgua.tex}

lalaa
çcvdzjmopfse
