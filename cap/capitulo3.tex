\chapter{O gerador de código}
\label{cap3}

Dada uma especificação na linguagem \TLA, contendo elementos da lógica TLA e da teoria de conjuntos, além de elementos sintáticos próprios, deseja-se obter uma definição equivalente em linguagem de programação. Equivalência para esse propósito é definida pela igualdade do conjunto de comportamentos permitidos. Isto é, todo comportamento especificado deve ser permitido na execução do código, e todo comportamento permitido pela execução do código deve ter sido especificado.

\section{Elixir}

Para esse propósito, a linguagem de programação escolhida para o código traduzido foi Elixir. As motivações são expostas abaixo por ordem de relevância na decisão:
\begin{enumerate}
  \item A concorrência é facilitada por ter seu código traduzido para \textit{bytecode} da máquina virtual do Erlang (BEAM). Suporte a concorrência é de extrema importância, já que \TLA foi criado para facilitar a especificação de sistemas concorrentes. É necessário que o código gerado seja capaz de refletir o sistema também nesse quesito.
  \item Uma linguagem funcional tende a se aproximar mais de definições matemáticas do que linguagens de outros paradigmas. Uma vez que a estrutura de \TLA foi construída principalmente no âmbito da matemática, a complexidade das traduções tende a ser menor para uma linguagem funcional.
  \item O alto nível de abstração da sintaxe de Elixir, que se inspira em Ruby e sua busca por código facilmente entendível, faz com o programador que trabalhar com o código gerado possa entendê-lo de forma mais simples e rápida do que seria com uma linguagem de baixo nível. Com isso, otimizações podem ser feitas com mais segurança, e a manutenabilidade do código é favorecida.
  \item A transparência de plataforma provida pela máquina virtual BEAM maximiza o número de ambientes aonde o código pode ser executado. Não seria de muito uso gerar um código para um ambiente específico, e uma máquina virtual permite que o código gerado seja \textit{Cross Plataform}.
  \item O seu código é aberto sobre a licença Apache 2.0, permitindo que o funcionamento de suas estruturas possa ser verificado a qualquer momento. Não seria possível garantir nenhuma correspondência do código gerado com a especificação se não fosse conhecida a execução gerada pelos operadores usados no código.

\end{enumerate}

Essa escolha vem de encontro com a finalidade de proporcionar um código modificável, de forma que o programador seja capaz de entender a correspondência e minimizando a diferença do nível de abstração no qual ele está programando.
