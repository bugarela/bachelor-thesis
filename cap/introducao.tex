\chapter{Introdução}
\label{introducao}

% A linguagem de especificação formal TLA+ (\textit{Temporal Logic of Actions}) foi desenvolvida por Leslie Lamport com o objetivo de escrever provas formais para sistemas concorrentes da maneira mais simples possível~\cite{tlahistory}. Para isso, o \textit{model checker} TLC foi desenvolvido, e mais recentemente o sistema de provas TLAPS (\textit{TLA Proof System}) \cite{tlaps2010}, que permite checar mecanicamente algumas provas e ainda está incompleto.

% Escrever uma especificação nessa linguagem possibilita verificar otmizações e encontrar potenciais problemas, incluindo \textit{bugs} e propiedades não satisfeitas. Esses benefícios foram reportados pela \textit{Amazon Web Services} ~\cite{amazon}, que afirma ter usado TLA+ em 10 sistemas complexos e, para cada um deles, ter encontrado \textit{bugs} ou adiquirido entendimento e confiança para implementar otimizações agressivas.

% As especificações escritas, contudo, não possuem nenhum vínculo com a implementação se não pelo entendimento do programador que as escreveu. Outras linguagens de especificação formal com objetivos semelhantes ao TLA+, como Z, B-Method e ASM, fornecem formas de gerar código a partir do modelo. Contudo, até a data da escrita desse texto, não foram encontrados geradores de código a partir de modelos escritos em TLA+, impossibilitando a conversão das especificações em linguagens de programação com garantia de correspondência.

% Com o programa especificado, validado e traduzido para linguagem de programação, é possível aplicá-lo diretamente em casos reais com a garantia de correspondência e, portanto, das propriedades verificadas; ou então melhorar a implementação para uma versão mais otimizada, mas que parte da mesma base - nesse caso, a garantia é reduzida, já que as mudanças não estavam representadas no modelo original. Observações sobre os benefícios da geração de código a partir de modelos de especificação formal já foram verificadas em trabalhos como o estudo de caso em \cite{Leonard2008}.

\section{Objetivos}

Esse trabalho é feito com a intenção de elaborar um método de tradução, através do mapeamento de estruturas e construtores, de especificações formais descritas em TLA+ para código em linguagem de programação com possibilidade de ser executado e modificado; assim como implementar um tradutor que aplique esse método.

\subsection{Objetivos Específicos}
\begin{itemize}
  \item Encontrar mapeamentos entre as estruturas de especificação em TLA+ e estruturas de linguagens de programação
  \item Implementar um gerador de código Elixir, com capacidade de fazer \textit{parsing} de especificações em TLA+ e aplicar os mapeamentos necessários.
\end{itemize}
