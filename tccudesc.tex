\NeedsTeXFormat{LaTeX2e}

\documentclass[a4paper,12pt]{pkg/monografia}
\usepackage[pdfpagelabels]{hyperref}
\usepackage{amsmath,amsthm,amsfonts,amssymb}
\usepackage[mathcal]{eucal}
\usepackage{latexsym}
%\usepackage[utf8]{inputenc}
\usepackage[brazil]{babel}
%\usepackage{bm}
\usepackage[alf]{pkg/abntex2cite}
\usepackage{url}
\usepackage{enumitem}
\usepackage{graphicx}
\usepackage{placeins}
%\usepackage{epstopdf}
\usepackage{multirow}
%\usepackage{fancyhdr}
\usepackage[FIGTOPCAP]{subfigure}
\usepackage{textcase}
\usepackage{tabularx}
\usepackage[table,xcdraw]{xcolor}
%\usepackage[portuguese,noend,ruled]{algorithm2e}
\usepackage{listings}
\usepackage[T1]{fontenc}
\usepackage{courier}
\usepackage[printonlyused]{acronym}
\usepackage{caption}
\usepackage{adjustbox}
\usepackage{floatrow}
\floatsetup[figure]{capposition=top}
\floatsetup[table]{capposition=top}
%\captionsetup{labelsep=endash}
%\usepackage{rotating}
%\usepackage{rotfloat}
%\usepackage{placeins}
\usepackage{fontspec}
\setmonofont{Inconsolata}
%\captionsetup[listing]{position=top}
\usepackage{makecell}
\usepackage{listings}

%-----------------------------------------------------------
\begin{document}

%----------------- Título e Dados do Autor -----------------
\titulo{Análise de arquiteturas de microsserviços empregados a jogos MMORPG voltada a otimização do uso de recursos de gerenciamento de mundos virtuais}
\autor{Marlon Henry Schweigert}
\nome{Marlon Henry}
\ultimonome{Schweigert}

%---------- Informe o Curso e Grau -------------------------
\bacharelado \curso{Ciência da Computação} \mes{Junho} \ano{2018}
\data{\today} % Data da aprovação
\cidade{Joinville}

%---------- Informações sobre a Institução -----------------
\instituicao{Universidade do Estado de Santa Catarina}
\sigla{UDESC} \unidadeacademica{Centro de Ciências Tecnológicas}

%-- Nomes do Orientador, 1o. Examinador e 2o. Examinador ---
\orientador{Charles Christian Miers}
\examinadorum{Débora Cabral Nazário}
\examinadordois{Guilherme Piêgas Koslovski}
%\examinadortres{}

%-- Títulos do Orientador, 1o. Examinador e 2o. Examinador -
\ttorientador{Doutor}
\ttexaminadorum{Doutora}
\ttexaminadordois{Doutor}
%\ttexaminadortres{}

%---------- Capa -------------------------------------------
\maketitle

%---------- Agradecimentos----------------------------------
%\agradecimento{Agradecimentos}
Pessoas incríveis fomentaram este trabalho. O Marcelo me mostrou Haskell, e isso
me direcionou à linha de pesquisa que me encantou. Anos depois, o Leandro me
perguntou se eu já tinha ouvido falar de \TLAA, e os videos do Lamport me fascinaram. Quando
tive a ideia de fazer um gerador de código, o Paulo estava do meu lado e foi o
primeiro a dizer sim, seguido dos outros super pesquisadores do grupo Função - e eles nunca me deixaram desistir da ideia.

Agradeço infinitamente a esse grupo e a mais um menino com quem programei minha
primeira sequência de Fibonacci, e que passou todo o
desenvolvimento do trabalho cerca de um metro à minha esquerda.
%\newpage

%---------- Epígrafe ---------------------------------------
%\begin{epigrafe}



%\end{epigrafe}

%---------- Resumo -----------------------------------------
\resumo{Resumo}
Especificar software traz garantias de segurança para o sistema e facilita modificações e otimizações sobre o programa especificado, o que é especialmente interessante para sistemas concorrentes ou distribuídos, onde há muitos comportamentos a se considerar. Especificações para esses sistemas podem ser escritas com a linguagem de especificação \TLAA, que é composta por definições próximas à matemática e ferramentas que permitem a verificação de propriedades descritas em lógica temporal. Este trabalho propõe um tradutor automático de especificações em \TLA para a linguagem de programação funcional e concorrente Elixir, permitindo a execução do sistema especificado.

\\
\noindent \textbf{Palavras-chaves:} Arquitetura de microsserviços, Desenvolvimento de jogos, Rede de jogos,
Jogos massivos, Otimização de recursos, Nuvens computacionais

%---------- Abstract ---------------------------------------
\resumo{Abstract}
Specifying software provides safety guarantees for the system and
facilitates modifications and optimizations on the specified program, which is specially interesting for concurrent or distributed systems, where there are many behaviors to be considered. Specifications for this systems can be written in the specification language \TLA, which is composed by definitions close to mathematics and tools that allow verification of temporal logic described properties. This work proposes an automatic translator of \TLA specifications to the funcional and concurrent programming language Elixir, allowing the execution of the specified system.

\\
\noindent \textbf{Keywords:} Cloud computing, Traffic characterization, Management network, Traffic monitoring system, Performance analysis, OpenStack.

%-----------------------------------------------------------

% SUMÁRIO, LISTA DE FIGURAS E LISTA DE TABELAS
%---------- Sumário ----------------------------------------
\tableofcontents
\thispagestyle{empty}

%---------- Lista de figuras -------------------------------
\listoffigures
\thispagestyle{empty}

%---------- Lista de tabelas -------------------------------
\listoftables
\thispagestyle{empty}

%---------- Lista de Abreviaturas --------------------------
\listofabbreviations{Lista de Abreviaturas}
\begin{acronym}[]
	\acro{amqp}[AMQP]{{\it Advanced Message Queuing Protocol}}
	\acro{api}[API]{{\it Application Programming Interface}}
  \acro{aws}[AWS]{{\it Amazon Web Services}}
	\acro{cli}[CLI]{{\it Command Line Interface}}
	\acro{crud}[CRUD]{{\it Create Read Update Delete}}
	\acro{cpu}[CPU]{{\it Central Processing Unit}}
	\acro{cs}[C/S]{{Cliente/Servidor}}
	\acro{ddos}[DDoS]{{\it Distributed Denial of Service}}
	\acro{fps}[FPS]{{\it First-person shooter}}
	\acro{http}[HTTP]{{\it Hypertext Transfer Protocol}}
	\acro{iaas}[IaaS]{{\it Infrastructure as a Service}}
	\acro{ide}[IDE]{{\it Integrated Development Environment}}
	\acro{idl}[IDL]{{\it Interface Description Language}}
  \acro{ids}[IDS]{{\it Intrusion Detection System}}
	\acro{json}[JSON]{{\it JavaScript Object Notation}}
	\acro{jwt}[JWT]{{\it JSON Web Token}}
	\acro{kvm}[KVM]{{\it Kernel-based Virtual Machine}}
	\acro{lan}[LAN]{{\it Local Area Network}}
  \acro{ldap}[LDAP]{{\it Lightweight Directory Access Protocol}}
	\acro{mhz}[MHz]{{\it Mega-hertz}}
	\acro{mmo}[MMO]{{\it Massively Multiplayer Online}}
	\acro{mmofps}[MMOFPS]{{\it Massively Multiplayer Online First-Person Shooter}}
	\acro{mmorpg}[MMORPG]{{\it Massively Multiplayer Online Role-Playing Game}}
	\acro{moba}[MOBA]{{\it Multiplayer Online Battle Arena}}
	\acro{mvc}[MVC]{{\it Model-View-Controller}}
	\acro{nist}[NIST]{{\it National Institute of Standards and Technology}}
	\acro{npcs}[NPCs]{{\it Non-Playable Characters}}
	\acro{ntsc}[NTSC]{{\it National Television System Committee}}
	\acro{p2p}[P2P]{{\it Peer-to-Peer}}
	\acro{pvp}[PvP]{{\it Player vs Player}}
	\acro{pvnpc}[PvNPCs]{{\it Player vs \ac{npcs}}}
	\acro{paas}[PaaS]{{\it Platform as a Service}}
	\acro{pov}[POF]{{\it Point of View}}
	\acro{qos}[QoS]{{\it Quality of Service}}
	\acro{ram}[RAM]{{\it Random Access Memory}}
	\acro{rest}[REST]{{\it Representational State Transfer}}
	\acro{rpc}[RPC]{{\it Remote Procedure Call}}
	\acro{rpg}[RPG]{{\it Role-Playing Game}}
	\acro{rts}[RTS]{{\it Real-Time Strategy}}
	\acro{sdn}[SDN]{{\it Software Defined Network}}
	\acro{saas}[SaaS]{{\it Software as a Service}}
	\acro{snmp}[SNMP]{{\it Simple Network Management Protocol}}
	\acro{tcp}[TCP]{{\it Transmission Control Protocol}}
	\acro{tps}[TPS]{{\it Third-person Shooter}}
	\acro{tia}[TIA]{{\it Television Interface Adapter}}
	\acro{udp}[UDP]{{\it User Datagram Protocol}}
	\acro{vlan}[VLAN]{{\it Virtual Local Area Network}}
	\acro{vm}[VM]{{\it Virtual Machine}}
	\acro{vpn}[VPN]{{\it Virtual Private Network}}
	\acro{wan}[WAN]{{\it Wide Area Network}}
	\acro{ws}[WS]{{\it Web Services}}
	\acro{xdr}[XDR]{{\it External Data Representation}}
	\acro{xml}[XML]{{\it Extensible Markup Language}}



	\acrodefplural{vpn}[VPNs]{{\it Virtual Private Networks}}
	\acrodefplural{vlan}[VLANs]{{\it Virtual Local Area Networks}}
	\acrodefplural{vm}[VMs]{{\it Virtual Machines}}
\end{acronym}

% Defining: \acro{acronym}[short name]{full name}
% Usaging:
% \ac{acronym}     -- writes the full name followed by the acronym in brackets; later calls will write only the acronym
% \acf{acronym}     -- writes the full name followed by the acronym in brackets
% \acs{acronym}     -- writes the short name only
% \acl{acronym}     -- writes the full name only
% Use p at the end of previous commands for plural form (e.g., \acp for the plural form of \ac)
% \acresetall        -- reset usage of all acronyms (i.e., \ac will print full name again)
% \acused                -- mark the acronym as used

\thispagestyle{empty}

%---------- Início do Conteúdo -----------------------------
\pagestyle{ruledheader}

\chapter{Introdução}
\label{introducao}

% A linguagem de especificação formal TLA+ (\textit{Temporal Logic of Actions}) foi desenvolvida por Leslie Lamport com o objetivo de escrever provas formais para sistemas concorrentes da maneira mais simples possível~\cite{tlahistory}. Para isso, o \textit{model checker} TLC foi desenvolvido, e mais recentemente o sistema de provas TLAPS (\textit{TLA Proof System}) \cite{tlaps2010}, que permite checar mecanicamente algumas provas e ainda está incompleto.

% Escrever uma especificação nessa linguagem possibilita verificar otmizações e encontrar potenciais problemas, incluindo \textit{bugs} e propiedades não satisfeitas. Esses benefícios foram reportados pela \textit{Amazon Web Services} ~\cite{amazon}, que afirma ter usado TLA+ em 10 sistemas complexos e, para cada um deles, ter encontrado \textit{bugs} ou adiquirido entendimento e confiança para implementar otimizações agressivas.

% As especificações escritas, contudo, não possuem nenhum vínculo com a implementação se não pelo entendimento do programador que as escreveu. Outras linguagens de especificação formal com objetivos semelhantes ao TLA+, como Z, B-Method e ASM, fornecem formas de gerar código a partir do modelo. Contudo, até a data da escrita desse texto, não foram encontrados geradores de código a partir de modelos escritos em TLA+, impossibilitando a conversão das especificações em linguagens de programação com garantia de correspondência.

% Com o programa especificado, validado e traduzido para linguagem de programação, é possível aplicá-lo diretamente em casos reais com a garantia de correspondência e, portanto, das propriedades verificadas; ou então melhorar a implementação para uma versão mais otimizada, mas que parte da mesma base - nesse caso, a garantia é reduzida, já que as mudanças não estavam representadas no modelo original. Observações sobre os benefícios da geração de código a partir de modelos de especificação formal já foram verificadas em trabalhos como o estudo de caso em \cite{Leonard2008}.

\section{Objetivos}

Esse trabalho é feito com a intenção de elaborar um método de tradução, através do mapeamento de estruturas e construtores, de especificações formais descritas em TLA+ para código em linguagem de programação com possibilidade de ser executado e modificado; assim como implementar um tradutor que aplique esse método.

\subsection{Objetivos Específicos}
\begin{itemize}
  \item Encontrar mapeamentos entre as estruturas de especificação em TLA+ e estruturas de linguagens de programação
  \item Implementar um gerador de código Elixir, com capacidade de fazer \textit{parsing} de especificações em TLA+ e aplicar os mapeamentos necessários.
\end{itemize}

\chapter{TLA\textsuperscript{+}}
\label{cap2}

TLA+ é uma linguagem de especificação de software, criada por Leslie Lamport [CITAR] voltada à modelagem de sistemas concorrentes. Ela se propõe a oferecer uma maneira mais simples de escrever um algoritmo, ao utilizar um nível de abstração acima do que há ao escrever código em uma linguagem de programação. Assim, ao programar, não é necessário atentar-se a detalhes de implementação, permitindo o foco no comportamento do algoritmo - e não das suas dependências.

As especificações são descritas em fórmulas matemáticas, com pequenas adaptações de sintaxe. Para facilitar a curva de aprendizado para engenheiros, foi criada a linguagem PlusCal, com uma sintaxe semelhante a linguagens de programação imperativas, e que traduz seus programas para TLA+. A linguagem PlusCal não permite especificar sistemas tão complexos quanto os que podem ser escritos diretamente em TLA+, mas, devido à tradução para a linguagem original, aproveita completamente as capacidades dela de verificação de propriedades.

O método de especificação é baseado em máquinas de estados e, sendo assim, a descrição de um modelo é composta por uma condição inicial, que determina os possíveis estados inciais, e por uma relação de transições, que determina os possíveis estados que podem suceder cada estado em uma execução. Dessa forma, o conjunto de comportamentos especificado é composto por todos os comportamentos cujo estado inicial satisfaz a condição inicial e todas as transições estão na relação.

Lamport destaca [HYPERBOOK] que as especificações deveriam ser sobre modelos de uma abstração do sistema, e não algo retirado do próprio sistema. Semelhante à planta de um edifício, a especificação pode ser consultada para obter informações sobre o edifício (ou programa) de forma mais conveniente, além de ser capaz de facilitar uma série de verificações e perceber problemas enquanto a mudança ainda não é inviavelmente custosa.

\input{JarrosDeAgua.tex}

lalaa
çcvdzjmopfse

\chapter{O gerador de código}
\label{cap3}

Dada uma especificação na linguagem \TLA, contendo elementos da lógica TLA e da teoria de conjuntos, além de elementos sintáticos próprios, deseja-se obter uma definição equivalente em linguagem de programação. Equivalência para esse propósito é definida pela igualdade do conjunto de comportamentos permitidos. Isto é, todo comportamento especificado deve ser permitido na execução do código, e todo comportamento permitido pela execução do código deve ter sido especificado.

\section{Elixir}

Para esse propósito, a linguagem de programação escolhida para o código traduzido foi Elixir. As motivações são expostas abaixo por ordem de relevância na decisão:
\begin{enumerate}
  \item A concorrência é facilitada por ter seu código traduzido para \textit{bytecode} da máquina virtual do Erlang (BEAM). Suporte a concorrência é de extrema importância, já que \TLA foi criado para facilitar a especificação de sistemas concorrentes. É necessário que o código gerado seja capaz de refletir o sistema também nesse quesito.
  \item Uma linguagem funcional tende a se aproximar mais de definições matemáticas do que linguagens de outros paradigmas. Uma vez que a estrutura de \TLA foi construída principalmente no âmbito da matemática, a complexidade das traduções tende a ser menor para uma linguagem funcional.
  \item O alto nível de abstração da sintaxe de Elixir, que se inspira em Ruby e sua busca por código facilmente entendível, faz com o programador que trabalhar com o código gerado possa entendê-lo de forma mais simples e rápida do que seria com uma linguagem de baixo nível. Com isso, otimizações podem ser feitas com mais segurança, e a manutenabilidade do código é favorecida.
  \item A transparência de plataforma provida pela máquina virtual BEAM maximiza o número de ambientes aonde o código pode ser executado. Não seria de muito uso gerar um código para um ambiente específico, e uma máquina virtual permite que o código gerado seja \textit{Cross Plataform}.
  \item O seu código é aberto sobre a licença Apache 2.0, permitindo que o funcionamento de suas estruturas possa ser verificado a qualquer momento. Não seria possível garantir nenhuma correspondência do código gerado com a especificação se não fosse conhecida a execução gerada pelos operadores usados no código.

\end{enumerate}

Essa escolha vem de encontro com a finalidade de proporcionar um código modificável, de forma que o programador seja capaz de entender a correspondência e minimizando a diferença do nível de abstração no qual ele está programando.

\input{cap/conclusao}

%Pessoas incríveis fomentaram este trabalho. O Marcelo me mostrou Haskell, e isso
me direcionou à linha de pesquisa que me encantou. Anos depois, o Leandro me
perguntou se eu já tinha ouvido falar de \TLAA, e os videos do Lamport me fascinaram. Quando
tive a ideia de fazer um gerador de código, o Paulo estava do meu lado e foi o
primeiro a dizer sim, seguido dos outros super pesquisadores do grupo Função - e eles nunca me deixaram desistir da ideia.

Agradeço infinitamente a esse grupo e a mais um menino com quem programei minha
primeira sequência de Fibonacci, e que passou todo o
desenvolvimento do trabalho cerca de um metro à minha esquerda.

%---------- Referências ------------------------------------
\renewcommand{\bibname}{Referências}
\bibliographystyle{pkg/abnt-alf}
\bibliography{refsTcc}

%---------- Apêndice ---------------------------------------
%\appendix
%\include{cap/apendice-publicacoes}

%-----------------------------------------------------------
\end{document}
