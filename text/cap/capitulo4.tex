\chapter{Considerações}

% protótipo, divulgação

Com a conclusão deste trabalho, consideram-se algumas contribuições previstas e
não previstas, bem como um conjunto de ideias para continuações e melhorias
da ferramenta implementada que ficaram de fora do escopo deste trabalho.

O desenvolvimento se dividiu em duas grandes etapas bastante distintas. A
primeira etapa envolveu, na maior parte, revisão do material bibliográfico e
didático disponível sobre TLA e \TLAA. O conhecimento obtido a partir desse
estudo foi exposto no Capítulo \ref{cap2}. Tal agregação de conhecimento e
sumarização de conceitos é deixada como contribuição didática, e espera-se
também que ajude na divulgação de \TLAA, especialmente no Brasil.

Essa etapa constrói a base de todo o trabalho - tanto para o texto apresentado,
quanto para o desenvolvimento da segunda etapa. Entender os conceitos de forma
precisa possibilitou a confiança necessária para as decisões de estruturação do
código gerado. A maior dificuldade no processo de aplicar esses conhecimentos,
que não estava prevista, foi o grau de interpretação necessário para estabelecer
um valor computacional em alguns aspectos da especificação, como foi
estabelecido o conceito de escolhas para estados onde mais de um próximo estado
satisfaz as ações possíveis.

A segunda etapa se deu pela definição dos mapeamentos através de regras de
tradução, assim como da implementação destas regras e validação através da
geração de código para os exemplos de especificações apresentados. Os resultados
obtidos são considerados suficientes para atender os objetivos no Capítulo
\ref{introducao}, não só mostrando a viabilidade da geração de código, mas
também entregando um tradutor automático e demonstrando o valor da execução do
código gerado.

Entende-se que o tradutor e as regras de tradução oferecidas também podem ter
valor didático, no sentido de que observar o código gerado pode ajudar no
entendimento do significado da estrutura que o gerou. O principal valor do
gerador de código é, contudo, visto como a capacidade de gerar protótipos de
validação e eventualmente bases iniciais para o desenvolvimento ou módulos a
serem acoplados 

Apesar de as regras de tradução apresentadas cobrirem todos os elementos
principais de \TLAA, o conjunto de especificações aceito por ele é bastante
restrito, já que a gramática de \TLA é muito vasta. Prevê-se que as
regras necessárias para traduzir o restante da linguagem sejam numerosas, porém
não muito complexas. Algumas traduções que já podem ser feitas a partir da
representação intermediária utilizada no tradutor não são feitas para todas suas
versões na linguagem de especificação, uma vez que \TLA possui vários açúcares
sintáticos e representa mais de um valor, predicado ou ação de várias maneiras
diferentes. Aceitar essas outras maneiras, para alguns casos, requer apenas
adições ao \textit{parser}.

No planejamento deste trabalho, esperava-se fornecer garantias para os
mapeamentos encontrados. Isso deixou de fazer sentido durante a execução por
considerar-se que o valor computacional de uma especificação é passível de
interpretação, e o código é gerado a partir da interpretação feita. Dessa forma,
o que mais se aproxima de uma garantia são evidências de que o código gerado
apresenta o valor resultante da interpretação, como feito ao mostrar o processo
de tradução e execução do código gerado para especificações.

A escolha da linguagem Elixir se mostrou muito benéfica, uma vez que ela permite
a comunicação do modelo com o oráculo e quaisquer outros processos de forma
simples e transparente. O código gerado pode se comunicar com um oráculo em
outra máquina, o que demostra a real possibilidade de usar o modelo gerado em um
sistema distribuído.

\section{Trabalhos futuros}

A continuação evidente deste trabalho é a adição de novos mapeamentos para que
um conjunto maior de especificações em \TLA seja traduzível. Outros caminhos de
exploração também foram observados, e julga-se que este trabalho possa receber
muitos complementos interessantes. Algumas reflexões são:
\begin{enumerate}
  \item Passos balbuciantes não são levados em conta durante toda a execução do
    modelo, não permitindo que uma ação que não altera o estado do modelo seja
    tomada. Esse tipo de ação tem o valor de uma espera, e acontece sempre que
    há uma situação de escolha, já que a escolha pode ser adiada por qualquer
    período de tempo. Quando o modelo é capaz de decidir a próxima ação, ele não
    oferece a oportunidade de alguma influência gerar um passo balbuciante, e
    irá alterar o estado imediatamente. Com o tradutor atual, seria possível
    definir um passo balbuciante como uma ação qualquer sem condição, fazendo
    com que uma escolha sempre seja necessária e
    todos os passos possam ser influenciados. Todavia, se julga interessante que
    houvessem formas de interromper o modelo a partir da interpretação de passos
    balbuciantes.

  \item No gerador proposto, se uma ação não definir o valor de uma variável no
    próximo estado - o que, na especificação, significaria que qualquer valor
    para aquela variável é válido - o estado será replicado sem aquela variável,
    o que geralmente resultará em erro em alguma ação futura. Uma melhoria para
    essa situação seria permitir que influências externas definam o valor dessa
    variável no próximo estado, já que ele pode ser qualquer.

  \item Muitas informações da especificação que não definem predicados ou ações
    são ignoradas ou precisam ser manualmente retiradas para o funcionamento do
    tradutor. Uma melhoria para a ferramenta seria a capacidade de interpretar
    essas informações e descartar adequadamente o que não tem valor para
    execução.

  \item Em vários exemplos de especificações, uma das invariantes checada é
    sobre os tipos das variáveis. Um trabalho bastante relevante seria a
    tradução dessas invariantes para assinaturas de tipo, de forma que o
    checador de tipos de Elixir mantenha as garantias de tipos, tornando as
    alterações de código mais seguras. Esse mesmo efeito pode ser obtido
    através da tradução de outras invariantes para grupos de testes
    compartilhados, de forma que o programador possa verificar essas invariantes
    em todos os testes que escrever - isto é, nos estados obtidos durante a
    execução dos testes implementados pelo programador.

  \item A especificação dos jarros de água estende a especificação dos inteiros,
    disponível por padrão na IDE de \TLAA. Essas especificações base poderiam
    ser traduzidas e, seus códigos gerados, disponibilizados juntamente com a
    ferramenta de tradução, de forma que ela carregue o código de uma
    especificação base sempre que ela for estendida por alguma outra
    implementação.
\end{enumerate}
